\documentclass[11pt]{article}
\usepackage{geometry}
\geometry{letterpaper}
\usepackage{graphicx}
\usepackage{amssymb}
\usepackage{amsmath}
%\usepackage{subfigure}
%\usepackage{float}
\usepackage{multirow}
%\usepackage{fullpage}
%\usepackage[compact]{titlesec} %Reduces the space around headings
\usepackage[parfill]{parskip}    % Activate to begin paragraphs with an empty line rather than an indent
%\usepackage{natbib} %For citations!

\title{ Design Document for COS 333 Project }
\author{Noah Apthorpe, Garrett Disco, Luke Paulsen, Jocelyn Tang, and Natalie Weires}

\begin{document}

\maketitle

\section{The Group}
\par Project name (tentative): ``Course Aggregator Tool" (CAT)
\par Project manager: Garrett Disco
\begin{tabbing}
Noah Apthorpe   \= apthorpe@princeton.edu \\
Garret Disco \> gdisco@princeton.edu \\
Luke Paulsen \> lpaulsen@princeton.edu \\
Jocelyn Tang \> jmtang@princeton.edu \\
Natalie Weires \> nweires@princeton.edu \\
\end{tabbing}

\section{Overview}
\par The Course Aggregator Tool is an online platform for viewing and querying Princeton course information available at various OIT and Registrar sites. This will help Princeton undergraduates make course selection and scheduling decisions, including long-term course planning. The site is different from other tools such as ICE and the Registrar site in that it will allow easy visualization of all available data on a course, make it easier to compare and rank courses, and provide historical data on older semesters of a course, but it is not designed to provide calendar-style scheduling the way ICE is.
\par CAT will provide a graphical interface that will make it easy for users to get a sense of what a course is like and select what information they want to see. It will also provide a way to compare two (or possibly more) courses side-by-side. The course data being displayed will include: professors, class size, class times, ratings in each category, written course reviews, P/D/F status, and textbook costs. It will be possible to search for, rank, and filter courses by any of these categories, either over the whole university or by department or distribution requirement. We will also include a system for rating course reviews ``up" or ``down" for its helpfulness.
\par CAT will be built with Python/Django and hosted on Amazon AWS. We will scrape course data periodically (e.g. once per semester) from the Registrar website and store it in a Mongo database. We are still deciding on what to use for the frontend, but it is likely to involve heavy use of Javascript/CSS and/or tools built on top of them.

\section{Functionality}
CAT has several different possible use cases, all of which involve a comparison of multiple courses.
\par Use case 1: A user wants to find a class fitting a particular kind of description (e.g. ``find a large LA class that is well-reviewed and offered after 11 AM"). The user logs in to CAT, which displays a prominent search box where all of these parameters can be put in (either a selection of fields, or a single text bar with smart interpretation, or both). Once the user has entered the search parameters, the best-matching courses are displayed, from which the user can select one or more courses to view the full data for.
\par Use case 2: A user is only considering a few courses (e.g. COS 333 and COS 402) and wants to compare their ratings in detail. The user opens both courses (by typing the course code into the search box, then opening another search box and repeating the process) and compares them side-by-side in terms of lecture rating, assignment rating, and overall course recommendation rating (all of these can be selected for display as part of the full data for the course).
\par Use case 3: A user is browsing courses and is considering what to take in future semesters. The user puts in very general search parameters (e.g. all COS classes) and looks at classes from the result list one at a time. When viewing courses, the user activates the ``timeline" option, which brings up past semesters of the course as well as the most recent semester. Then user can then view how course ratings, class size, readings, and so on have changed over time.

\par Core functionality that will be required:
\begin{itemize}
\item Search for and rank courses by one or more categories
\item Display full data for one or more courses from search results
\item Add additional boxes to search for and view courses in
\item Extend a single semester to a timeline for that course
\item Choose which data to display for courses
\end{itemize}

\section{Design}
\par We have decided to use mongoDB for the database because a lot of the data is irregular - for example, classes have different numbers of sections, multiple professors, etc. Each class is a document with fields for each feature: professor, sections, ratings and reviews, pdf status, distribution category, sample readings, grading breakdown, class number, pre-requisites, course description.
\par Our primary source of data is the registrar. We plan to scrape the course offerings website as well as the ratings and reviews, which come from applyweb.com.
\par We plan to use the Django framework to structure the website. It provides a powerful and well-organized system for accessing the database and sending information to the client. The Django models system will be used as the interface for database access, and the Django view functions will be used as the interface for interacting with the client side.
\par We plan to use some kind of Javascript (most likely jQuery) for the frontend. This will let us provide an intuitive and interactive user interface, while accessing server-side data efficiently using Ajax requests.

\section{Milestones}
\begin{enumerate}
\item Set up site on AWS (i.e. have ``Hello World" working)
\item Set up backend on AWS, send data back and forth
\item Get in contact with OIT / Registrar
\item Define database structure (all categories we will need)
\item Scrape one semester worth of course data
\item Display single semester data in bare-bones format
\item Scrape multiple semesters and display all at once
\item Search, filter, and rank by any category
\item Decide on frontend design specs for course view (move earlier?)
\item Decide on frontend design specs for search view
\par [No particular order from here on out]
\item Implement frontend design specs for search view
\item Implement frontend design specs for course view
\item Implement ranking system for course reviews
\item Implement textual analysis of course reviews
\item Implement future semester predictions
\item Implement CAS login wall
\item Deploy for alpha test
\item Deploy for beta test
\item Deploy for demo
\end{enumerate}

\section{Risks}
Risks:
\begin{itemize}
\item Getting access to data: We may have to meet with OIT or the registrar to scrape the ratings and reviews for courses, since this data requires login information.
\item Changing course numbers: In the case that departments change course numbers, we need to be able to track this for each semester. We can do this by comparing numbers of classes that have the same names over the course of the semester.
\end{itemize}

\section{Timeline}
Timeline is being maintained but is not yet posted online.

\end{document}